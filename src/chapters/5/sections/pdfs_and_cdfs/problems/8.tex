(a)
Let $F$ be the CDF of the Beta distribution with parameters $a=3$, $b=2$.

Due to the properties of the CDF, $F$ must be 0 for $x \le 0$ and 1 for $x \ge 1$.

For $0 < x < 1$:

\begin{equation*}
\begin{split}
F(x)
& = \int_0^x f(t) dt = \int_0^x 12 t^2 (1-t) dt = 12 \int_0^x t^2 dt - 12 \int_0^x t^3 dt \\
& = 12 \left( \frac{x^3}{3} - \frac{x^4}{4} \right) \\
& = 4 x^3 - 3 x^4
\end{split}
\end{equation*}

Then:

$$
F(x) =
\begin{cases}
0           &\text{ , for } x \le 0 \\
x^3(4 - 3x) &\text{ , for } 0 < x < 1 \\
1           &\text{ , for } x \ge 1
\end{cases}
$$


(b)
\begin{equation*}
\begin{split}
P(0 < X < 1/2)
& = F(1/2) - F(0) = \left( \frac{1}{2} \right)^3 \left( 4 - \frac{3}{2} \right) - 0 \\
& = \frac{5}{16}
\end{split}
\end{equation*}


(c)
The mean of $X$ can be calculated by the definition of expectation:

\begin{equation*}
\begin{split}
E(X)
& = \int_0^1 x f(x) dx = \int_0^1 12 x^3 (1-x) dx \\
& = 12 \int_0^1 x^3 dx - 12 \int_0^1 x^4 dx = 12 \left( \frac{1}{4} - \frac{1}{5} \right) \\
& = \frac{3}{5}
\end{split}
\end{equation*}


By the definition of variance:

$$
\mathrm{Var}(X) = E(X^2) - (EX)^2
$$

Let's calculate the second moment of $X$ by LOTUS:

\begin{equation*}
\begin{split}
E(X^2)
& = \int_0^1 x^2 f(x) dx = \int_0^1 12 x^4 (1 - x) dx \\
& = 12 \int_0^1 x^4 dx - 12 \int_0^1 x^5 dx = 12 \left( \frac{1}{5} - \frac{1}{6} \right) \\
& = \frac{2}{5}
\end{split}
\end{equation*}


Substituting this value into the definition of variance:

$$
\mathrm{Var}(X) = \frac{2}{5} - \left( \frac{3}{5} \right)^2 = \frac{1}{25}
$$
