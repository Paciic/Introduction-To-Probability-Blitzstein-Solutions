\begin{enumerate}[label=(\alph*)]

    \item $\frac{1}{2^{|C|}}$

    \item For each element of $C$ we have four options for whether this element is in $A$ and/or $B$ and each option has equal probability of occurring.
          An element $x \in C$ is in $A \subseteq B$ if and only if ($x \in B \text{ and } x \in A$) or ($x \in B \text{ and } x \notin A$) or ($x \notin B \text{ and } \notin A$), i.e., 3 times out of 4.
          Thus, the probability that $x \in A \subseteq B$ is $3/4$ and $P(A \subseteq B) = (3/4)^{100}$.

          Another way to see this is by using the naive definition of probability.
          The sample space consists of 100 binary pairs where 1 in the 1st slot of the $i$-th pair indicates that the $i$-th element of $C$ is in $A$ and 1 in the 2nd slot indicates that the element is in $B$.
          Hence, $|S| = 4^{100}$.
          The number of elements in the set $X$ of outcomes corresponding to $A \subseteq B$ can be counted as
          $$|X| = \sum_{i=0}^{100} \binom{100}{i}2^i = 3^{100}.$$
          The binomial coefficient accounts for the number of $i$-element subsets $B$ of $C$ and $2^i$ is the number of $i$-element subsets $A$ of $B$.
          This gives $P(A \subseteq B) = |X| / |S| = (3/4)^{100}.$

    \item Let $p$ be a randomly selected person from $C$ sampled without
          replacement.

          $P(p \in A \cup p \in B) = \frac{1}{2} + \frac{1}{2} - \frac{1}{4} = \frac{3}
              {4}$.

          $P(A \cup B = C) = (P(p \in A \cup p \in B))^{|C|} = \left(\frac{3}{4}\right)^
              {|C|}.$
\end{enumerate}