\begin{enumerate}[label=(\alph*)]
\item Let $X$ be the number of people that play the same opponent in both
rounds. Let $I_{j}$ be the indicator variable that person $j$ plays against the
same opponent twice. $P(I_{j}=1) = \frac{1}{99}$. Then, $\text{E}(X) = \sum_
{j=1}^{100}P(I_{j}=1) = 100/99$.

\item There is a strong dependence between trials. For instance, if we know that
the first $50$ players played the same opponent twice, then all of the players
played the same opponents twice. Moreover, knowing each of the $I_{i}$ gives us perfect information about one other $I$ - they are strongly pairwise dependent.

\item Consider the $50$ pairs that played each other in round one. Let $I_{j}$
be the indicator variable for pair $j$ playing each other again in the second
round. $P(I_{j}=1) = \frac{1}{99}$. Then, the expected number of pairs that
play the same opponent twice is $\text{E}(Z) = \frac{50}{99} \approx \frac{1}{2}$.

We can approximate the number of pairs that play against one another in both
rounds with $Z \sim \text{Poiss}(\frac{1}{2})$. Note that $X=2Z$. $P(X = 0) \approx P(Z =
0) = e^{-\frac{1}{2}} \approx 0.6$.

$P(X = 2) \approx P(Z = 1) = \frac{(\frac{1}{2})^{1}e^{-\frac{1}{2}}}{1!}
\approx 0.32$.

Note that the approximation in part C is more accurate - the independence of the same pairs playing against each other is much stronger than the independence of individuals who play the same opponent. Knowing that the players in Game 1 of round 2 played against each other in round 1 gives us very little information about whether players in any other games also played against each other. Whereas, knowing that Player 1 in round 2 plays against the same player (say, player 71) guarantees that we know that that player 71 also plays against the same player.  
\end{enumerate}
